% We hope this article will encourage more social scientists to apply the methods of spatial modeling, and combine them with regression models already common in social sciences.  Spatial methods are already in use within social sciences, and we've highlighted a few key examples.  However, the methods remain limited to specialty areas and applications.

% It is often the case that the input data is more important than the technical details of the model, and this is certain to be true also of geographic disaggregation of demographic data.  For this reason, we believe that future research should focus on testing the relationship between geographical features and demographic attributes of interest.  This may often require field work or clever research designs that could not be applied at a broad geographic scale.  Small focused research efforts with limited geographic scope, however, can yield valuable and rigorously tested results that can support design decisions of models with broad or even global coverage.  For this research to be most effective it should test with data sources that are most valuable to models intended to have broad coverage.  Examples include land use data, climatology, satellite data, and crowdsourced data like OpenStreetMap.  A better understanding of how data sources like these relate to demographic variables will be valuable when designing models to spatially dissaggregate the data that characterizes human populations.

% The article also discussed extrapolation of RK models and potential ethical issues that may arise when doing so.  Future work will explore these technical and ethical concerns by considering extrapolations of the model presented in this article to other locations.
% , but deserve to be re-presented in a fashion that makes clear the solutions that are appropriate for each class of problems typical of social science models.

% cut-down/out
% Spatial modeling terminology often reflects the contexts where a model was developed; selecting the right model is often made more difficult by different understandings of what the models are designed to do.  For example, indicator kriging is designed to eliminate skew and outliers in a univariate spatial model by discretizing continuous data into a series of indicators that identify cases above and below a given value; this is especially valuable when searching for areas of space above a threshold--eg finding profitable seams of ore.  All discretization indicators are co-kriged to estimate the averages or totals over an area--eg average ore grade or total extractable ore.  Social science indicators may also be discretizations of continuous variables, eg a poverty line indicator, but equally often they are fundamentally categorical variables, such as biological sex.  Determining whether so-called indicator kriging is appropriate for a social science indicator can be a confusing task.  Goovaerts has possibly done more than any other to review and compare the performance of different kriging models on social science data, especially health data.

% Maps of social science data, in particular indicators, may be difficult to interpret.  Health data often has a strong spatial component because it is linked to underlying environmental variables.  These models are straightforward to interpret because they model a relationship with a physical entity.  A map of poverty, however, represents data with strong social components, and spatial patterns may be more arbitrary.  In some areas of the world mansions abut slums, creating sharp arbitrary ``poverty boundaries'' that a spatial model would have difficulty reproducing.  The maps also typically represent risk, and while social scientists are familiar with risk as a scientific concept, the mapped risk may be only loosely linked to spatial causes.  Poverty may have a spatial component, and one might improve their lot by leaving impoverished areas, but in general the causes of poverty are social and not spatial, so that simply moving away from a location wouldn't necessarily alter the risk of someone being in poverty.  This reality of social data is fundamentally different from the assumptions we make when mapping data, and this can leave the maps fraught, requiring that we think carefully about our interpretations of mapped social data.


%%% Local Variables:
%%% mode: latex
%%% TeX-master: "main"
%%% End:
