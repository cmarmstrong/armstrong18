% A good start.  There's some repetition that could be fixed with a little rewriting.  Some parts are like an annotated bibliography, and could be summarized.  There are many good points to embellish upon with similar research.  Do you think the drawing of administrative boundaries across the developing world during European empirialism is another example of artificial boundaries coming to influence the population within in a way that makes the boundary more real?

Sociologists have in recent years devoted considerable study, especially from a policy perspective, on the problem of administrative boundaries. % citation needed, maybe those found by Marissa on policy (wong, 2015, orford, 2017)
Administrative boundaries are held to be essentially arbitrary concepts which are mapped onto human societies and result in the application of boundary-congruent policies which do not reflect the reality of the variation in the human populations of the area including their transboundary presence.

However, a second component of boundaries exist. Policies, when applied, having impacts to transboundary human populations even when they are not calibrated for the commonality of the human population across the boundary. These real impacts can create a long-term legacy which actually creates differences and distincts which did not previously exist. This hypothesis of a deep institutional impact on social and cultural behaviors and attitudes would in general present missed covariates and endogenous causation challenges which would make this assertion difficult to defend.

However, Peisakhin, in ``In History's Shadow: Do Formal Institutions Leave a Cultural Legacy?'' \cite{peisakhin13} uses the division of the western Ukrainian population between the Russian Empire and Austrian Empire in the 19th century as a test-case. His analysis of cultural impacts on political and social behaviors serves as a powerful support for the idea that ``formal institutions might be capable of producing lasting cultural legacies'' \cite{peisakhin13}.  % paraphrase instead of a quote

This possibility suggests that the impact of policy may be recursive. Policies that are calibrated by arbitrary borders may not merely ignore the needs of transboundary social groups, but in fact produce new, artificially induced distinctions and differences in those social groups which persist for decades or centuries after the border itself has been eliminated by the course of historical geopolitics.

Perhaps the most extreme example is by Acemoglu, Daron, Simon Johnson and James Robinson, ``Social Structure and Development: A Legacy of the Holocaust in Russia'' \cite{acemoglu11}.  In it, the authors present evidence proposing that long-term changes in economic and political outcomes in modern Russia can be directly correlated to the legacy of the Holocaust. There is no more extreme government policy in human history than the genocide perpetuated by the Third Reich's occupational authorities in the Soviet Union, and so it provides a definitive check for the hypothesis that long-term economic and political outcomes may be correlated with it even though the "policy" and the regime which perpetuated it are long gone and the regions which suffered under the holocaust have been under shared administration with those which did not for more than 70 years. % a good theme.  might replace the third reich with "holocaust in the Soviet Union" to be more economical with words.

The Second World War in all its tragedy provides a fertile field for these comparisons. The war was directly responsible for the partition of Germany from 1945 - 1990, and this partition serves as the next study, by Alberto Alesina and Nicola Fuchs-Schundeln, "Good-Bye Lenin (or Not?): The Effect of Communism on People's Preferences" \cite{alesina07}. The same effect of policy on a homogenous society divided by radically different policies and then restored to a single government and single administrative policy is analyzed. Alesina and Fuchs-Schundeln even propose a time estimate of 20 - 40 years for the effect to completely vanish, however, they caution that it explicitly assumes a linear trajectory. Acemoglu and Peisakhin \cite{peisakhin13}, \cite{acemoglu11} suggest that a linear trajectory is not the case, or is only the case when a government attempts serious, substantive efforts to eliminate the differences with policies which are explicitly tailored to the now nonexistent administrative boundaries (!) in a reverse of the typical case of policies following current administrative boundaries and ignoring social differences.  % great points.  the trajectory descriptions are hard to follow, a picture might help

While several additional examples could be drawn from the consequences of the Second World War, it is sensible to move further afield to consider the impact of different events on non-European societies. In ``History, Institutions, and Economic Performance: The Legacy of Colonial Land Tenure Systems in India'' by Abhijit Banerjee and Lakshmi Iyer \cite{banerjee05}, the authors reference Rafael La Porta's argument that historically being colonized by Britain rather than another colonial power resulted in the best long-term economic performance. Following the same research path \cite{banerjee05}, the authors consider whether the system for collecting land revenue in India resulted in demonstratable changes in health and education, as well as agricultural investment.

It is worthwhile here to briefly cover the situation of colonial India. Colonial India consisted of two forms of sovereignty, the directly Imperial ruled provinces (and previously the ``Company Raj'' before 1856), and the protectorates or vassal states (the ``Princely States'') of the Viceroyalty. The analysis by Banerjee and Iyer focuses on the Company Raj/Imperial Provinces \cite{banerjee05}, where the collection of taxes was essentially either on a collective village basis (the Northwest), an individual basis (different areas of the country), or by Zamindari Estates. Zamindars were essentially hereditary tax collectors who owned the right to collect taxes on land and by extension held a kind of quasi-ownership in which they were able to dispossess tenants whom they could not collect the government's taxes from. % the first sentence seems repetitive.  didn't the last paragraph already raise the case of colonial India?

Banerjee and Iyer find that areas with Zamindari land tenure lag considerably behind the other regions of modern India, even though modern India is now organized into States based on linguistic groups which cross the land-tenure boundaries and incorporate areas now sharing common administrative policy which once had different kinds of land tenure. This provides a strong argument in support of the notion that boundaries in administrative policy alone can, even decades after they are ended, cause major shifts in development across a host of important metrics. 


\subsection{A tale of two cities}

Perhaps one of the strangest examples of arbitrary differences in administrative boundaries and their impact on the development of a region is that between the north German cities of Bremerhaven and Cuxhaven, located in the Waddensea region. Both cities share the unusual distinction of having been under the control of another city, the Imperial Free Cities of Bremen and Hamburg respectively, until 1939 during Nazi administrative “rationalizations” of Germany. Essentially an Imperial Free City was a City which was also a Federal State of Germany in its various incarnations before that point; Bremerhaven and Cuxhaven were the deep sea ports of their respective trading cities and of considerable economic importance.

For complicated, but in administrative terms essentially arbitrary reasons, after WW2 the allied administrative powers restored Bremerhaven to the control of Bremen, now a city-state federal entity within West Germany. Hamburg was also in West Germany, as was Cuxhaven, however, Cuxhaven was not restored to the control of Hamburg. This creates a unique distinction as Cuxhaven was now a conventionally organized city within a territorial state and Bremerhaven a subject of the Federal City-State of Bremen, with the difference in administrative status being totally arbitrary.

As a result, Cuxhaven is predominantly services oriented, with a cruise-ship terminal, museums, vacation services and a fishing industry. Bremerhaven has conversely been developed into one of the foremost deepwater ports in the world. The demographic projections for the 21st century show the staggering difference: As the economy shifts toward the service sector, Cuxhaven has been more able to keep up with the changing world and is expected to shrink by only 2.5\% in population between 2000 and 2020. Bremerhaven, devoted to increasingly automated heavy industry in an economy where it is less relevant, is predicted to shrink by 23\% in population over the same period (The Waddensea Region, a Socio-Economic Analysis by Prognos AG, June, 2004).

In looking at the relationship between Cuxhaven and Bremerhaven on a Development Labour Force / Development GDP Axis, we see that Cuxhaven saw its labor force remain the same from 1992 – 2000 while its GDP grew by 25\%; Bremerhaven, chained to particular industry sectors by zoning controlled by its larger ‘owner’ of Bremen, saw a decline of 5\% in population and a growth in GDP of only 5\% (The Waddensea Region, a Socio-Economic Analysis by Prognos AG, June, 2004).

One can see from this comparison that the completely arbitrary retention of Bremerhaven as an isolated enclave administratively controlled by Bremen versus the attachment of Cuxhaven to the state of Lower Saxony rather than the retention of its position as an enclave of Hamburg as resulted in a radically different outcome for both cities. Of course, the population has also changed dramatically; in this, the population of Bremerhaven reached 120,000 before declining due to aggressive expansion pursued by Bremen in terms of annexations and development. In the case of Cuxhaven, the population plateaued at 55,000 and then began to decline earlier, but also more slowly, than in the case of Bremerhaven.


%%% Local Variables:
%%% mode: latex
%%% TeX-master: "main"
%%% End:
